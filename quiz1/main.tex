\documentclass{homework}

\title{Quiz 1}

\author{Yair Nahum (id: 034462796)\\And\\Daniel Teitelman (id: 207734088)}

\begin{document}

\maketitle


(i) 

when considering a diagonal matrix, we need to approximate that all pixels illuminations $x$ are roughly the same and we treat all the same w/o relating to the actual pixels values.

Since all cameras get the same scene inputs, the averages and max values of inputs are the same. We just need to convert the values to the same dynamic range for each color component.

Meaning, we assume $r(\lambda , x)$ is constant relative to $x \hspace{5pt} ( r(\lambda,x)\approx r(\lambda) )$.

According to the per color channel sensor response calculation:
$$I_c=\int_{\lambda} \Phi(\lambda)f_c(\lambda)d\lambda=\int_{\lambda} e(\lambda)r(\lambda)f_c(\lambda)d\lambda$$
When,\\
$e(\lambda) \equiv \text{ the illuminant spectrum} $\\
$r(\lambda) \equiv \text{ material reflectance} $\\
$\Phi(\lambda) = e(\lambda)r(\lambda) \equiv \text{ light SPD (Spectral Power Distribution) that is received back from the surface} $\\
$f_c(\lambda) \equiv \text{ sensor SSF (Spectral Sensitivity Function) per color channel } c$


\end{document}