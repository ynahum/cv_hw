\documentclass{homework}

\title{Quiz 1}

\author{Yair Nahum (id: 034462796)\\And\\Daniel Teitelman (id: 207734088)}

\begin{document}

\maketitle


(i) 

when considering a diagonal matrix, we need to approximate that all pixels illuminations $x$ are roughly the same and we treat all the same w/o relating to the actual pixels values.

Since all cameras get the same scene inputs, the averages and max values of inputs are the same. We just need to convert the values to the same dynamic range for each color component.

Meaning, we assume $r(\lambda , x)$ is constant relative to $x \hspace{5pt} ( r(\lambda,x)\approx r(\lambda) )$.

According to the per color channel sensor response calculation:
$$I_c=\int_{\lambda} \Phi(\lambda)f_c(\lambda)d\lambda=\int_{\lambda} e(\lambda)r(\lambda)f_c(\lambda)d\lambda$$
When:\\
$e(\lambda) \equiv \text{ the illuminant spectrum} $\\
$r(\lambda) \equiv \text{ material reflectance} $\\
$\Phi(\lambda) = e(\lambda)r(\lambda) \equiv \text{ light SPD (Spectral Power Distribution) that is received back from the surface} $\\
$f_c(\lambda) \equiv \text{ sensor SSF (Spectral Sensitivity Function) per color channel } c$

We define a diagonal transformation matrix from camera i to camera j as follows:

$$T_{i\rightarrow j} \equiv
\begin{pmatrix} 
    S_b & 0 & 0 \\ 
    0 & S_g & 0 \\ 
    0 & 0 & S_r 
\end{pmatrix}$$

It converts a pixel vector 3x1 $\begin{pmatrix} R_i\\G_i\\B_i \end{pmatrix}$ as follows:
$$\begin{pmatrix} R_j\\G_j\\B_j \end{pmatrix} = T_{i\rightarrow j}=
\begin{pmatrix} 
    S_b & 0 & 0 \\ 
    0 & S_g & 0 \\ 
    0 & 0 & S_r 
\end{pmatrix}\begin{pmatrix} R_i\\G_i\\B_i \end{pmatrix}$$

Since the $e(\lambda)$ is assumed to be built as spikes (of 0.8, but it doesn't really matter as all cameras get the same), we can approximate $I_c$ for each camera.

Note that the spikes are at the mean of the Gaussians except for camera 3.

$$I_{1,b}=0.8*1, I_{1,g}=0.8*1, I_{1,r}=0.8*1$$
$$I_{2,b}=0.8*0.5, I_{2,g}=0.8*1, I_{2,r}=0.8*1$$
$$I_{3,b}=0.8*1, I_{3,g}=0.8*1, I_{3,r}=0.8*\exp{(-\frac{(650-600)^2}{2*(70)^2})}$$

Therefore, the diagonal transformation matrices are as follows:

$$T_{2\rightarrow1}= \begin{pmatrix} 
    2 & 0 & 0 \\ 
    0 & 1 & 0 \\ 
    0 & 0 & 1 
\end{pmatrix}$$
$$T_{3\rightarrow1}= \begin{pmatrix} 
    1 & 0 & 0 \\ 
    0 & 1 & 0 \\ 
    0 & 0 & \exp{(\frac{(650-600)^2}{2*(70)^2})} 
\end{pmatrix}$$

(ii)

If we don't assume the naive assumption that all material reflectances are the same. Meaning, we have $r(\lambda,x)$.

Then, we need to find a 3x3 transformation matrix that minimizes the difference (squared) between pixels perceived by camera 1 and our camera.

We define $X_i$ as the $n$ pixels matrix of camera $i$ when $X_i\in \mathbb{R}^{3xn}$\\
Thus, we have the following optimization problem:
$$min_{T_{i\rightarrow j}}{\begin{Vmatrix}X_j-T_{i\rightarrow j}X_i\end{Vmatrix}}_F^2$$

We can solve it directly and analytically using SVD decompositions of $X_i$ and $X_j$ and calculating gradient relative to transformation matrix and compare to 0.



\end{document}